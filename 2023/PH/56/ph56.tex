\iffalse
\let\negmedspace\undefined
\let\negthickspace\undefined
\documentclass[journal,12pt,onecolumn]{IEEEtran}
\usepackage{cite}
\usepackage{amsmath,amssymb,amsfonts,amsthm}

\usepackage{graphicx}
\usepackage{textcomp}
\usepackage{xcolor}
\usepackage{txfonts}
\usepackage{listings}
\usepackage{enumitem}
\usepackage{mathtools}
\usepackage{gensymb}
\usepackage[breaklinks=true]{hyperref}
\usepackage{tkz-euclide} % loads  TikZ and tkz-base
\usepackage{listings}
\usepackage{gvv}
\usepackage{booktabs}

%
%\usepackage{setspace}
%\usepackage{gensymb}
%\doublespacing
%\singlespacing

%\usepackage{graphicx}
%\usepackage{amssymb}
%\usepackage{relsize}
%\usepackage[cmex10]{amsmath}
%\usepackage{amsthm}
%\interdisplaylinepenalty=2500
%\savesymbol{iint}
%\usepackage{txfonts}
%\restoresymbol{TXF}{iint}
%\usepackage{wasysym}
%\usepackage{amsthm}
%\usepackage{iithtlc}
%\usepackage{mathrsfs}
%\usepackage{txfonts}
%\usepackage{stfloats}
%\usepackage{bm}
%\usepackage{cite}
%\usepackage{cases}
%\usepackage{subfig}
%\usepackage{xtab}
%\usepackage{longtable}
%\usepackage{multirow}

%\usepackage{algpseudocode}
%\usepackage{enumitem}
%\usepackage{mathtools}
%\usepackage{tikz}
%\usepackage{circuitikz}
%\usepackage{verbatim}
%\usepackage{tfrupee}
%\usepackage{stmaryrd}
%\usetkzobj{all}
%    \usepackage{color}                                            %%
%    \usepackage{array}                                            %%
%    \usepackage{longtable}                                        %%
%    \usepackage{calc}                                             %%
%    \usepackage{multirow}                                         %%
%    \usepackage{hhline}                                           %%
%    \usepackage{ifthen}                                           %%
  %optionally (for landscape tables embedded in another document): %%
%    \usepackage{lscape}     
%\usepackage{multicol}
%\usepackage{chngcntr}
%\usepackage{enumerate}

%\usepackage{wasysym}
%\documentclass[conference]{IEEEtran}
%\IEEEoverridecommandlockouts
% The preceding line is only needed to identify funding in the first footnote. If that is unneeded, please comment it out.

\newtheorem{theorem}{Theorem}[section]
\newtheorem{problem}{Problem}
\newtheorem{proposition}{Proposition}[section]
\newtheorem{lemma}{Lemma}[section]
\newtheorem{corollary}[theorem]{Corollary}
\newtheorem{example}{Example}[section]
\newtheorem{definition}[problem]{Definition}
%\newtheorem{thm}{Theorem}[section] 
%\newtheorem{defn}[thm]{Definition}
%\newtheorem{algorithm}{Algorithm}[section]
%\newtheorem{cor}{Corollary}
\newcommand{\BEQA}{\begin{eqnarray}}
\newcommand{\EEQA}{\end{eqnarray}}
\newcommand{\define}{\stackrel{\triangle}{=}}
\theoremstyle{remark}
\newtheorem{rem}{Remark}

%\bibliographystyle{ieeetr}
\begin{document}
%

\bibliographystyle{IEEEtran}


\vspace{3cm}

\title{
%	\logo{
Gate Question

\large{EE:1205 Signals and Systems}

Indian Institute of Technology, Hyderabad
%	}
}
\author{Abhey Garg

EE23BTECH11202
}	


% make the title area
\maketitle


%\tableofcontents

\bigskip

\renewcommand{\thefigure}{\arabic{figure}}
\renewcommand{\thetable}{\arabic{table}}
\renewcommand{\theequation}{\arabic{equation}}

\section{Question GATE PH 56}
Consider the complex function
\[ f(z) = \frac{z^{2}\sin z}{(z-\pi)^4} \]
At \( z = \pi \), which of the following options is (are) correct?
\begin{enumerate}[label=\textbf{\arabic*.}, font=\bfseries, align=left]
    \item[(A)] The order of the pole is 4 
    \item[(B)] The order of the pole is 3 
    \item[(C)] The residue at the pole is \( \frac{\pi}{6} \)
    \item[(D)] The residue at the pole is \( \frac{2\pi}{3} \)
\end{enumerate}
\hfill (GATE PH 2023)
\section{Solution}
\fi
\setlength{\arrayrulewidth}{0.3mm}
\setlength{\tabcolsep}{20pt}
\renewcommand{\arraystretch}{1.3}



\begin{tabular}{|c|c|c|}
\hline

Parameter& Description & Remarks\\
\hline
$V_c\brak{0^{-}}$ & Voltage across capacitor when t$<$0 & 20V\\
\hline
$i_L \brak{0^-}$ & current across inductor when t$<$0 & $0.2 $  \\
\hline
$i_L \brak{0^+}$ & current across inductor when t$>$0 & $0.2 $\\
\hline
$C$ & Capacitance & 0.01F\\
\hline
$L$ & Inductance & 1H\\
\hline

\end{tabular}


\begin{enumerate}[label=\textbf{\arabic*.}, font=\bfseries, align=left]
\item[(a)]
As the power of $(z-\pi) $ in denominator is 4 , so the order of the pole is 4.
\item[(b)]
\begin{align}
\text{Res}(f, \pi) &= \frac{1}{(m-1)!} \frac{d^{m-1}}{dz^{m-1}} \left[ (z-\pi)^m f(z) \right] \Bigg|_{z=\pi} \\
\text{Res}(f,\pi) &= \frac{1}{3!} \frac{d^3}{dz^3} \left[ (z-\pi)^4 \frac{z^2 \sin z}{(z-\pi)^4} \right] \Bigg|_{z=\pi} \\
\text{Res}(f,\pi) &= \frac{1}{3!} \frac{d^3}{dz^3} z^2 \sin z \Bigg|_{z=\pi} \\
&= \frac{1}{3!} (6\cos z -6z\sin z -z^2 \cos z) \Bigg|_{z=\pi} \\
\end{align}


Since \( \sin(\pi) = 0 \) and \( \cos(\pi) = -1 \), this simplifies to:


\begin{align}
\text{Res}(f,\pi) &= \frac{\pi^2-6}{3!}  = \frac{\pi^2 - 6}{6}
\end{align}
\end{enumerate}
%\end{document}
