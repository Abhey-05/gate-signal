%\iffalse
\let\negmedspace\undefined
\let\negthickspace\undefined
\documentclass[journal,12pt,onecolumn]{IEEEtran}
\usepackage{cite}
\usepackage{amsmath,amssymb,amsfonts,amsthm}
%\usepackage{algorithmic}
\usepackage{graphicx}
\usepackage{textcomp}
\usepackage{array}
\usepackage{xcolor}
\usepackage{txfonts}
\usepackage{listings}
\usepackage{enumitem}
\usepackage{mathtools}
\usepackage{gensymb}
\usepackage[breaklinks=true]{hyperref}
\usepackage{tkz-euclide} % loads  TikZ and tkz-base
\usepackage{listings}
\usepackage{float}
\usepackage{bm}



\newtheorem{theorem}{Theorem}[section]
\newtheorem{problem}{Problem}
\newtheorem{proposition}{Proposition}[section]
\newtheorem{lemma}{Lemma}[section]
\newtheorem{corollary}[theorem]{Corollary}
\newtheorem{example}{Example}[section]
\newtheorem{definition}[problem]{Definition}
%\newtheorem{thm}{Theorem}[section] 
%\newtheorem{defn}[thm]{Definition}
%\newtheorem{algorithm}{Algorithm}[section]
%\newtheorem{cor}{Corollary}
\newcommand{\BEQA}{\begin{eqnarray}}
\newcommand{\EEQA}{\end{eqnarray}}
\newcommand{\define}{\stackrel{\triangle}{=}}
\theoremstyle{remark}
\newtheorem{rem}{Remark}
%\bibliographystyle{ieeetr}
\begin{document}
%
\providecommand{\pr}[1]{\ensuremath{\Pr\left(#1\right)}}
\providecommand{\prt}[2]{\ensuremath{p_{#1}^{\left(#2\right)} }}        % own macro for this question
\providecommand{\qfunc}[1]{\ensuremath{Q\left(#1\right)}}
\providecommand{\sbrak}[1]{\ensuremath{{}\left[#1\right]}}
\providecommand{\lsbrak}[1]{\ensuremath{{}\left[#1\right.}}
\providecommand{\rsbrak}[1]{\ensuremath{{}\left.#1\right]}}
\providecommand{\brak}[1]{\ensuremath{\left(#1\right)}}
\providecommand{\lbrak}[1]{\ensuremath{\left(#1\right.}}
\providecommand{\rbrak}[1]{\ensuremath{\left.#1\right)}}
\providecommand{\cbrak}[1]{\ensuremath{\left\{#1\right\}}}
\providecommand{\lcbrak}[1]{\ensuremath{\left\{#1\right.}}
\providecommand{\rcbrak}[1]{\ensuremath{\left.#1\right\}}}
\newcommand{\sgn}{\mathop{\mathrm{sgn}}}
\providecommand{\abs}[1]{\left\vert#1\right\vert}
\providecommand{\res}[1]{\Res\displaylimits_{#1}} 
\providecommand{\norm}[1]{\left\lVert#1\right\rVert}
%\providecommand{\norm}[1]{\lVert#1\rVert}
\providecommand{\mtx}[1]{\mathbf{#1}}
\providecommand{\mean}[1]{E\left[ #1 \right]}
\providecommand{\cond}[2]{#1\middle|#2}
\providecommand{\fourier}{\overset{\mathcal{F}}{ \rightleftharpoons}}
\newenvironment{amatrix}[1]{%
  \left(\begin{array}{@{}*{#1}{c}|c@{}}
}{%
  \end{array}\right)
}
%\providecommand{\hilbert}{\overset{\mathcal{H}}{ \rightleftharpoons}}
%\providecommand{\system}{\overset{\mathcal{H}}{ \longleftrightarrow}}
	%\newcommand{\solution}[2]{\textbf{Solution:}{#1}}
\newcommand{\solution}{\noindent \textbf{Solution: }}
\newcommand{\cosec}{\,\text{cosec}\,}
\providecommand{\dec}[2]{\ensuremath{\overset{#1}{\underset{#2}{\gtrless}}}}
\newcommand{\myvec}[1]{\ensuremath{\begin{pmatrix}#1\end{pmatrix}}}
\newcommand{\mydet}[1]{\ensuremath{\begin{vmatrix}#1\end{vmatrix}}}
\newcommand{\myaugvec}[2]{\ensuremath{\begin{amatrix}{#1}#2\end{amatrix}}}
\providecommand{\rank}{\text{rank}}
\providecommand{\pr}[1]{\ensuremath{\Pr\left(#1\right)}}
\providecommand{\qfunc}[1]{\ensuremath{Q\left(#1\right)}}
	\newcommand*{\permcomb}[4][0mu]{{{}^{#3}\mkern#1#2_{#4}}}
\newcommand*{\perm}[1][-3mu]{\permcomb[#1]{P}}
\newcommand*{\comb}[1][-1mu]{\permcomb[#1]{C}}
\providecommand{\qfunc}[1]{\ensuremath{Q\left(#1\right)}}
\providecommand{\gauss}[2]{\mathcal{N}\ensuremath{\left(#1,#2\right)}}
\providecommand{\diff}[2]{\ensuremath{\frac{d{#1}}{d{#2}}}}
\providecommand{\myceil}[1]{\left \lceil #1 \right \rceil }
\newcommand\figref{Fig.~\ref}
\newcommand\tabref{Table~\ref}
\newcommand{\sinc}{\,\text{sinc}\,}
\newcommand{\rect}{\,\text{rect}\,}
%%
%	%\newcommand{\solution}[2]{\textbf{Solution:}{#1}}
%\newcommand{\solution}{\noindent \textbf{Solution: }}
%\newcommand{\cosec}{\,\text{cosec}\,}
%\numberwithin{equation}{section}
%\numberwithin{equation}{subsection}
%\numberwithin{problem}{section}
%\numberwithin{definition}{section}
%\makeatletter
%\@addtoreset{figure}{problem}
%\makeatother

%\let\StandardTheFigure\thefigure
\let\vec\mathbf

\bibliographystyle{IEEEtran}





\bigskip

%\renewcommand{\thefigure}{\theenumi}
%\renewcommand{\thetable}{\theenumi}
%\renewcommand{\theequation}{\theenumi}

Q: The state equation of a second order system is \\
$ \dot{{x}}(t) = A{x}(t)$, \quad ${x}(0)$ is the initial condition. \\
Suppose $\lambda_1$ and $\lambda_2$ are two distinct eigenvalues of $A$, and $\nu_1$ and $\nu_2$ are the corresponding eigenvectors. For constants $\alpha_1$ and $\alpha_2$, the solution, ${x}(t)$, of the state equation is \\
\begin{enumerate}[label=(\Alph*)]
\item $\sum_{i=1}^{2} \alpha_ie^{\lambda_it}\bf{\nu}_i$
\item $\sum_{i=1}^{2} \alpha_ie^{2\lambda_it}\bf{\nu}_i$
\item $\sum_{i=1}^{2} \alpha_ie^{3\lambda_it}\bf{\nu}_i$
\item $\sum_{i=1}^{2} \alpha_ie^{4\lambda_it}\bf{\nu}_i$
\end{enumerate}
\hfill{GATE EC 2023}


\solution \\
%$ \dot{\bm{x}}(t) = A\bm{x}(t)$ \\
%If $\lambda$ is the eigen value of matrix A then $ \dot{\bm{x}}(t) = \lambda\bm{x}(t)$ \\
%As there are 2 eigen values $\lambda_1$ and $\lambda_2$ of matrix A, the solution of state equation will be, \\
%$x(t) = \sum_{i=1}^{2} \alpha_ie^{\lambda_it}\bf{\nu}_i$ \\
%Hence, the correct option is (A). \\
\begin{table}[!ht]
    \centering
        \begin{tabular}{|c|c|c|} 
    \hline
    \textbf{Variable} & \textbf{Description} & \textbf{Value} \\
    \hline
    $x(t)$ & input function & none \\
    \hline
    $y(t)$ & output function & $\sin(\pi t)$ \\
    \hline
    $H(s)$ & Transfer-function & $\frac{s-\pi}{s+\pi}$ \\
    \hline
\end{tabular}

    \caption{input parameters}
    \label{tab:gate23EC43.1}
\end{table}

\textbf{Theories and Proofs:} \\
\begin{align}
A\vec{y} &= \lambda \vec{y} 
\end{align}
Eigen values of inverse of a matrix is reciprocal of eigen value of the given matrix\\
\begin{align}
A^{-1}A\vec{y} &= A^{-1}\lambda \vec{y} \\ \implies
\vec{y} &= \lambda A^{-1}\vec{y}  \\ \implies
A^{-1} \vec{y} &= \frac{1}{\lambda}\vec{y} \label{eq:gate23EC43.1}
\end{align}
Eigen value of a matrix shifts by the same amount as that of the matrix.\\
\begin{align}
(A - \sigma I)\vec{y} &= A\vec{y} - \sigma I\vec{y} \\
&= \lambda \vec{y} - \sigma \vec{y} \\
&= (\lambda - \sigma) \vec{y} \label{eq:gate23EC43.2}
\end{align}
\textbf{Sol:} \\
Using Laplace transform: \\
Given Equation:
\begin{align}
\dot{{x}}(t) &= A{x}(t) \\
\frac{d{x}(t)}{dt} &= A{x}(t) 
\end{align}
Taking Laplace Transform:
\begin{align}
\mathcal{L}\brak{\frac{d{x}(t)}{dt}} &= \mathcal{L}\brak{A{x}(t)} \\
sX(s) - {x}(0) &= AX(s) \\
(sI - A)X(s) &= x(0) \\
X(s) &= (sI - A)^{-1}{x}(0) \label{eq:gate23EC43.3}
\end{align}
From \tabref{tab:gate23EC43.1}, we can write $x(0)$ in terms of two linearly independent variables as 
\begin{align}
    x(0) &= \alpha_1v_1 + \alpha_2v_2 \\
    &= \sum_{i=1}^{2}\alpha_iv_i \label{eq:gate23EC43.4}
\end{align}
From \eqref{eq:gate23EC43.3}, \eqref{eq:gate23EC43.4}

\begin{align}
 X(s) &= (sI - A)^{-1}\brak{\sum_{i=1}^{2}\alpha_iv_i} \\
 &= \sum_{i=1}^{2}(sI - A)^{-1}\alpha_iv_i
\end{align}
From \eqref{eq:gate23EC43.1}, \eqref{eq:gate23EC43.2}
\begin{align}
X(s) &=  \sum_{i=1}^{2}\frac{1}{s-\lambda_i}\brak{\alpha_iv_i} 
\end{align}
Now take inverse Laplace Transform
\begin{align}
\mathcal{L}^{-1}\brak{X(s)} &= \mathcal{L}^{-1}\brak{\sum_{i=1}^{2}\frac{1}{s-\lambda_i}\brak{\alpha_iv_i}} \\
{x}(t) &= \sum_{i=1}^{2} e^{\lambda_it}\brak{\alpha_iv_i} 
\end{align}

Hence the answer is option (A).
\end{document}
