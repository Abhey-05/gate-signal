\iffalse
\let\negthickspace\undefined
\documentclass[journal,12pt,twocolumn]{IEEEtran}
\usepackage{cite}
\usepackage{amsmath,amssymb,amsfonts,amsthm}
\usepackage{algorithmic}
\usepackage{graphicx}
\usepackage{textcomp}
\usepackage{xcolor}
\usepackage{txfonts}
\usepackage{listings}
\usepackage{enumitem}
\usepackage{mathtools}
\usepackage{gensymb}
\usepackage{comment}
\usepackage[breaklinks=true]{hyperref}
\usepackage{tkz-euclide} 
\usepackage{listings}
\usepackage{gvv}                                        
\def\inputGnumericTable{}                                 
\usepackage[latin1]{inputenc}                                
\usepackage{color}                                            
\usepackage{array}                                            
\usepackage{longtable}                                       
\usepackage{calc}                                             
\usepackage{multirow}                                         
\usepackage{hhline}                                           
\usepackage{ifthen}                                           
\usepackage{lscape}
\setlength{\arrayrulewidth}{0.5mm}
\setlength{\tabcolsep}{18pt}
\renewcommand{\arraystretch}{1.5}
\newtheorem{theorem}{Theorem}[section]
\newtheorem{problem}{Problem}
\newtheorem{proposition}{Proposition}[section]
\newtheorem{lemma}{Lemma}[section]
\newtheorem{corollary}[theorem]{Corollary}
\newtheorem{example}{Example}[section]
\newtheorem{definition}[problem]{Definition}
\newcommand{\BEQA}{\begin{eqnarray}}
\newcommand{\EEQA}{\end{eqnarray}}
\newcommand{\define}{\stackrel{\triangle}{=}}
\theoremstyle{remark}
\newtheorem{rem}{Remark}
\begin{document}
\title{GATE-2023 (EC)  Q.13}
\author{EE23BTECH11051-Rajnil Malviya}
\date{February 2024}
\maketitle
\subsection*{\textit{Question :-}}
Let $ w ^{4} = 16j $. Which of the following can not be the value of w?\\\\
(A)   $2e^\frac{j2 \pi}{8}$\\
(B)   $2e^\frac{j \pi}{8}$\\
(C)   $2e^\frac{j5 \pi}{8}$\\
(D)   $2e^\frac{j9 \pi}{8}$\\
Solution:-
\fi
\begin{align}
  \brak{w^4}^{\frac{1}{4}} &= \brak{16 j}^{\frac{1}{4}}
\end{align}
Using De-Moivre's theorem for $n^{th}$ root of $w$ ,
\begin{align}
    w &=  2j^{\frac{1}{4}}\\
    e^{j\theta} &= \cos{\theta}+j\sin{\theta}\label{eq:gate23ec13rajmal}
    \end{align}
    Using equation \eqref{eq:gate23ec13rajmal} and put $\theta =\brak {2n+1} \frac{\pi}{2}$
    \begin{align}
     w &=2e^{ [j \brak {2n+1} \frac{\pi}{2}] \frac{1}{4}}
\end{align}
For different values of n ,
\begin{align}
    n&=0 \implies w =2e^{ \frac{j \pi}{8}}\\
     n&=2 \implies w =2e^{ \frac{j 5\pi}{8}}\\
      n&=4 \implies w =2e^{ \frac{j 9\pi}{8}}
\end{align}
Ans . (A) $2e^\frac{j2 \pi}{8}$
%\end{document}
